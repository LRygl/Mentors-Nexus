\documentclass{article}
\usepackage[utf8]{inputenc}
\usepackage{fancyhdr}
\usepackage{nameref}
\usepackage[top=2cm,bottom=2cm,left=2cm,right=2cm]{geometry}
\usepackage{lastpage}
\usepackage{glossaries}
\usepackage{graphicx}
\usepackage{array}
\usepackage{pdflscape}
\usepackage{afterpage}
\usepackage{capt-of}% or use the larger `caption` package
\usepackage{longtable}
\usepackage{hyperref}
\usepackage{plantuml}
\usepackage{xstring}
\usepackage{catchfile}
\usepackage{color, colortbl}
\usepackage{import}
\usepackage{listings}
\usepackage{seqsplit}
\usepackage{tikz}
\usetikzlibrary{calc}
\usepackage{anyfontsize}
\usepackage{sectsty}
\usepackage{rest-api}

%_SET_GRAPHICS_PATH
\graphicspath{ {./Images/} }

%_IMPORT_CONFIGURATION_AND_STYLES
\import{Configuration/}{imports.sty}
\import{Configuration/}{configuration.sty}


\definecolor{getLightBlue}{HTML}{ECF6FF}
\definecolor{getLightGreen}{HTML}{EBF7F4}
\definecolor{getLightOrange}{HTML}{FEF4EB}
\definecolor{getLightRed}{HTML}{FBEEEE}




\newcommand{\GET}{getLightBlue}
\newcommand{\POST}{getLightGreen}
\newcommand{\PUT}{getLightOrange}
\newcommand{\DELETE}{getLightRed}

\newenvironment{ChangeToA3Paper}{%
\newlength\oldtextwidth
\oldtextwidth=\the\textwidth
\newpage
\pageaiii % Change page to A3: 297mm x 420mm
\setlength{\pdfpagewidth}{\paperwidth} % Change the pdf page
\setlength{\pdfpageheight}{\paperheight} % Change the pdf height
\setlength{\textwidth}{\the\paperwidth-\the\spinemargin-\the\foremargin} % Change the textwidth
\begin{adjustwidth}{0cm}{-87mm} % Enlarge the right margin by: 297 - 210 = 87mm
}
{
\end{adjustwidth}
\newpage
\pageaiv % Change page to A4: 210mm x 297mm
\setlength{\pdfpagewidth}{\paperwidth} % Change the pdf page
\setlength{\pdfpageheight}{\paperheight} % Change the pdf height
\textwidth=\the\oldtextwidth
}


\begin{document}
    \import{Cover/}{CoverPage.tex}
    \tableofcontents
    \listoffigures
    \listoftables
    \lstlistoflistings
\newpage
   

\section{Version Control}
\begin{small}
    \begin{longtable}[h]{|L{3cm}|L{2cm}|L{3cm}|L{4cm}|} 
        \hline
        \rowcolor{Gray}
        Version&Date&User&Change Note\\
        \hline
        \endhead% header material
        ver1.0.0-3cbac7&14.03.2022&Lubomír Rýgl&Initial version\\        \hline
        &&&\\        \hline
        &&&\\        \hline        
        &&&\\        \hline
        &&&\\        \hline   
        &&&\\        \hline
        &&&\\        \hline   
        &&&\\        \hline
        &&&\\        \hline   
        &&&\\        \hline
        &&&\\        \hline   
        &&&\\        \hline
        &&&\\        \hline   
        &&&\\        \hline
        &&&\\        \hline   
        &&&\\        \hline
        &&&\\        \hline   
        \caption{Document version history}
    \end{longtable}
\end{small}

\section{Application Overview}
Mentors Nexus is a software web application used by Mentors.cz to provide its users access to educational materials online. The application back-end is running on Java 18 with SpringBoot Framework and the front-end will be a Angular application. For more detail about the availabilty and architacture - please see the documentation Architecture Overview. The following sections describe the actions that can be taken by users interfacing with the application.
\subsection[short]{Application}

User
Course
Lesson
Path

USER ACTION
\begin{itemize}
    \itemsep 0em 
    \item Register Account - Allow user to register a new account with an email address
    \item Log-In to Account
    \item Update User
    \item Delete User
    \item Subscribe to Course
    \item Purchase Course
    \item Watch Lesson
    \item Finish Lesson
    \item Finish Course    
\end{itemize}

COURSE ACTION
\begin{itemize}
    \itemsep 0em 
    \item Create Course
    \item Edit Course
    \item Delete Course
\end{itemize}

LESSON ACTION
\begin{itemize}
    \itemsep 0em 
    \item Create Lesson
    \item Edit Lesson
    \item Delete Lesson
    \item Assign Lesson to Course
\end{itemize}

LEARNING PATH ACTION
\begin{itemize}
    \itemsep 0em 
    \item Create Learingn PATH
    \item Edit Learingn PATH
    \item Delete Learning Path
    \item Assign Course to Learning Path
\end{itemize}


\section{Architecture Overview}
\subsection{General Architecture}
asa
\subsubsection{Customer}
Basic user with its information and priviledges
\subsubsection{Course}
Course consists of lessons
\subsubsection{Lesson}
Actual lesson with its content
\subsubsection{Course-Category}
Each course has a category assigned

\subsection{AWS Architecture}
Application shall be build as a three tier application leveraging AWS Routing and Auto-Scalig groups. Whole solution is designed as HA.

\subsection{Endpoint Overview}
List of endpoints exposed to the frontend application available for request processing

\begin{lstlisting}[language=Java, caption=Backend Application base path]
        https://mentors.cz/api/v01/
\end{lstlisting}

\begin{table}[ht]
    \centering
    \begin{tabular}{|L{2cm}|L{2cm}|L{2cm}|L{4cm}|L{6cm}|}
        \hline
        \rowcolor{Gray}
        Category&Allowed Method&Type&Endpoint&Description \\ \hline
        %CONTENT%
        \rowcolor{\GET}
        User&GET&PUBLIC&/public-status&Shows application status \\ \hline
        \rowcolor{\POST}
        User&POST&PUBLIC&/user/resetPassword&Reset user password \\ \hline
        \rowcolor{\PUT}
        User&PUT&PROTECTED&/user/update&Register a new user  \\ \hline
        \rowcolor{\DELETE}
        User&DELETE&PROTECTED&/user/delete&Register a new user  \\ \hline

        \rowcolor{\GET}
        Category&GET&PUBLIC&/public-status&Shows application status \\ \hline
        \rowcolor{\POST}
        Category&GPOSTET&PUBLIC&/user/resetPassword&Reset user password \\ \hline
        \rowcolor{\PUT}
        Category&PUT&PROTECTED&/user/update&Register a new user  \\ \hline
        \rowcolor{\DELETE}
        Category&DELETE&PROTECTED&/user/delete&Register a new user  \\ \hline

        \rowcolor{\GET}
        Course&GET&PUBLIC&/public-status&Shows application status \\ \hline
        \rowcolor{\POST}
        Course&POST&PUBLIC&/user/resetPassword&Reset user password \\ \hline
        \rowcolor{\PUT}
        Course&PUT&PROTECTED&/user/update&Register a new user  \\ \hline
        \rowcolor{\DELETE}
        Course&DELETE&PROTECTED&/user/delete&Register a new user  \\ \hline

        \rowcolor{\GET}
        Lesson&GET&PUBLIC&/public-status&Shows application status \\ \hline
        \rowcolor{\POST}
        Lesson&POST&PUBLIC&/user/resetPassword&Reset user password \\ \hline
        \rowcolor{\PUT}
        Lesson&PUT&PROTECTED&/user/update&Register a new user  \\ \hline
        \rowcolor{\DELETE}
        Lesson&DELETE&PROTECTED&/user/delete&Register a new user  \\ \hline

    \end{tabular}
    \caption{List of available endpoints}
\end{table}









\section{Endpoint Definitions}
\import{Endpoints/User}{user.tex}


\section{Database Design}


\begin{lstlisting}[language=SQL, caption=Format XML for MAC calculation example]
SELECT * FROM application_users as;
\end{lstlisting}


\end{document}