\documentclass{article}
\usepackage[utf8]{inputenc}
\usepackage{fancyhdr}
\usepackage{nameref}
\usepackage[top=2cm,bottom=2cm,left=2cm,right=2cm]{geometry}
\usepackage{lastpage}
\usepackage{glossaries}
\usepackage{graphicx}
\usepackage{array}
\usepackage{pdflscape}
\usepackage{afterpage}
\usepackage{capt-of}% or use the larger `caption` package
\usepackage{longtable}
\usepackage{hyperref}
\usepackage{plantuml}
\usepackage{xstring}
\usepackage{catchfile}
\usepackage{color, colortbl}
\usepackage{import}
\usepackage{listings}
\usepackage{seqsplit}
\usepackage{tikz}
\usetikzlibrary{calc}
\usepackage{anyfontsize}
\usepackage{sectsty}
\usepackage{rest-api}

%_SET_GRAPHICS_PATH
\graphicspath{ {./Images/} }

%_IMPORT_CONFIGURATION_AND_STYLES
\import{Configuration/}{imports.sty}
\import{Configuration/}{configuration.sty}


\newenvironment{ChangeToA3Paper}{%
\newlength\oldtextwidth
\oldtextwidth=\the\textwidth
\newpage
\pageaiii % Change page to A3: 297mm x 420mm
\setlength{\pdfpagewidth}{\paperwidth} % Change the pdf page
\setlength{\pdfpageheight}{\paperheight} % Change the pdf height
\setlength{\textwidth}{\the\paperwidth-\the\spinemargin-\the\foremargin} % Change the textwidth
\begin{adjustwidth}{0cm}{-87mm} % Enlarge the right margin by: 297 - 210 = 87mm
}
{
\end{adjustwidth}
\newpage
\pageaiv % Change page to A4: 210mm x 297mm
\setlength{\pdfpagewidth}{\paperwidth} % Change the pdf page
\setlength{\pdfpageheight}{\paperheight} % Change the pdf height
\textwidth=\the\oldtextwidth
}


\begin{document}
    \import{Cover/}{CoverPage.tex}
    \tableofcontents
    \listoffigures
    \listoftables
    \lstlistoflistings
\newpage
   

\section{Version Control}
\begin{small}
    \begin{longtable}[h]{|L{3cm}|L{2cm}|L{3cm}|L{4cm}|} 
        \hline
        \rowcolor{Gray}
        Version&Date&User&Change Note\\
        \hline
        \endhead% header material
        ver1.0.0-3cbac7&14.03.2022&Lubomír Rýgl&Initial version\\        \hline
        &&&\\        \hline
        &&&\\        \hline        
        &&&\\        \hline
        &&&\\        \hline   
        &&&\\        \hline
        &&&\\        \hline   
        &&&\\        \hline
        &&&\\        \hline   
        &&&\\        \hline
        &&&\\        \hline   
        &&&\\        \hline
        &&&\\        \hline   
        &&&\\        \hline
        &&&\\        \hline   
        &&&\\        \hline
        &&&\\        \hline   
        \caption{Document version history}
    \end{longtable}
\end{small}

\section{Application Overview}
Mentors Nexus is a software web application used by Mentors.cz to provide its users access to educational materials online. The application back-end is running on Java 18 with SpringBoot Framework and the front-end will be a Angular application. For more detail about the availabilty and architacture - please see the documentation Architecture Overview. The following sections describe the actions that can be taken by users interfacing with the application.
\subsection[short]{Application}

USER ACTION
\begin{itemize}
    \itemsep 0em 
    \item Register Account 
    \item Log-In to Account
    \item Update User
    \item Delete User
    \item Subscribe to Course
    \item Purchase Course
    \item Watch Lesson
    \item Finish Lesson
    \item Finish Course    
\end{itemize}

COURSE ACTION
\begin{itemize}
    \itemsep 0em 
    \item Create Course
    \item Edit Course
    \item Delete Course
\end{itemize}

LESSON ACTION
\begin{itemize}
    \itemsep 0em 
    \item Create Lesson
    \item Edit Lesson
    \item Delete Lesson
    \item Assign Lesson to Course
\end{itemize}

LEARNING PATH ACTION
\begin{itemize}
    \itemsep 0em 
    \item Create Learingn PATH
    \item Edit Learingn PATH
    \item Delete Learning Path
    \item Assign Course to Learning Path
\end{itemize}


\section{Architecture Overview}
\subsection{General Architecture}

\subsection{AWS Architecture}
Application shall be build as a three tier application leveraging AWS

\subsection{Endpoint Overview}
List of endpoints exposed to the frontend application available for request processing

\begin{lstlisting}[language=Java, caption=Backend Application base path]
        https://mentors.cz/api/v01/
\end{lstlisting}

\begin{table}[ht]
    \centering
    \begin{tabular}{|L{2cm}|L{2cm}|L{6cm}|L{6cm}|}
        \hline
        \rowcolor{Gray}
        Allowed Method&Type&Endpoint&Description \\ \hline
        %CONTENT%
        \rowcolor{ApiBlue}
        GET&PUBLIC&/public-status&Shows application status \\ \hline

        \rowcolor{ApiBlue}
        GET&PROTECTED&/status&Shows application status \\ \hline
        \rowcolor{ApiBlue}
        GET&PUBLIC&/user/resetPassword&Reset user password \\ \hline

        \rowcolor{ApiGreen}
        POST&PUBLIC&/user/register&Register a new user  \\ \hline
        \rowcolor{ApiGreen}
        POST&PUBLIC&/user/ľogin&Log-In an existing user \\ \hline
        \rowcolor{ApiGreen}
        POST&PROTECTED&/user/add&Add new user manually - requires a administrator permissions to perform this acction. Endpoint is not publicly accessible  \\ \hline
        \rowcolor{ApiGreen}
        POST&PUBLIC&/user/register&Register a new user  \\ \hline

    \end{tabular}
    \caption{List of available endpoints}
\end{table}









\section{Endpoint Definitions}
%https://github.com/MusApfel/latex-describe-rest-api
\begin{apiRoute}{get}{/api/v01/user/find/\{id\}}{Get user record by user Id}
	
	\begin{routeParameter}
		\routeParamItem{id}{id of storage}
	\end{routeParameter}
	\begin{routeResponse}{application/json}
		\begin{routeResponseItem}{200}{ok}
			\begin{routeResponseItemBody}
{     
	"id": 867654678,
	"name" : "Apfelmus",
	"count" : 25
}
			\end{routeResponseItemBody}
		\end{routeResponseItem}
		\begin{routeResponseItem}{404}{error: storage not found}
			\begin{routeResponseItemBody}
{
	"message": "storage with id '11' not found!"
}
			\end{routeResponseItemBody}
		\end{routeResponseItem}
	\end{routeResponse}
	
\end{apiRoute}



\begin{apiRoute}{post}{/api/storage/}{create a new storage}
	\begin{routeParameter}
		\noRouteParameter{no parameter}
	\end{routeParameter}
	\begin{routeRequest}{application/json}
		\begin{routeRequestBody}
{
	"name" : "Apfelmus",
	"count" : 25
}
		\end{routeRequestBody}
	\end{routeRequest}
	\begin{routeResponse}{application/json}
		\begin{routeResponseItem}{200}{ok}
			\begin{routeResponseItemBody}
{     
	"id": 867654678,
	"name" : "Apfelmus",
	"count" : 25
}
			\end{routeResponseItemBody}
		\end{routeResponseItem}
	\end{routeResponse}
\end{apiRoute}



\section{Database Design}


\begin{lstlisting}[language=Java, caption=Format XML for MAC calculation example]
        
    /**
    * Returns part of XML string from which MAC should be generated.
    * param xml XML String
    * param macElementName XML element name from which the MAC should be generated
    */
   fun getXmlPartForMacVerification(xml: String, macElementName: String): String {
       val macStartElement = "<$macElementName>"
       val macEndElement = "</$macElementName>"
       if (xml.contains(macStartElement) && xml.contains(macEndElement)) {
           val macData = xml.substring(
               xml.indexOf(macStartElement),
               xml.indexOf(macEndElement) + macElementName.length + TAG_OFFSET
           )
           val macString = macData.replace(">[\\s]*<".toRegex(), "><")
           log.debug("MAC string length: {}", macString.length)
           return macString
       } else {
           throw ParsingErrorException("Missing element $macElementName for MAC.")
       }
   }
\end{lstlisting}


\end{document}